\documentclass[a4paper,12pt]{article}

% Pakete
\usepackage[utf8]{inputenc}
\usepackage[T1]{fontenc}
\usepackage[ngerman]{babel}
\usepackage{graphicx}
\usepackage{hyperref}
\usepackage{titlesec}

% Einstellungen
\titleformat{\section}[block]{\bfseries\Large}{\thesection.}{1em}{}
\titleformat{\subsection}[block]{\bfseries\large}{\thesubsection.}{1em}{}

\begin{document}

% Titel
\begin{titlepage}
    \centering
    \vspace*{2cm}
    {\bfseries\Large Werkzeuge für das wissenschaftliche Arbeiten\\}
    \vspace{0.5cm}
    \textbf{Python for Machine Learning and Data Science}\\
    \vspace{1cm}
    Abgabe: 15.12.2023\\
    \vspace{2cm}
    \hrule \vspace{0.5cm}
\end{titlepage}

% Inhaltsverzeichnis
\tableofcontents
\newpage

% Projektaufgabe
\section{Projektaufgabe}
In dieser Aufgabe beschäftigen wir uns mit der Objektorientierung in Python. Der Fokus liegt auf der Implementierung einer Klasse, dabei nutzen wir insbesondere auch Magic Methods.

\begin{figure}[h!]
    \centering
    \includegraphics[width=0.8\textwidth]{../diagram/classes_files.svg}
    \caption{Darstellung der Klassenbeziehungen.}
    \label{fig:classes}
\end{figure}

\subsection{Einleitung}
Ein Datensatz besteht aus mehreren Daten, ein einzelnes Datum wird durch ein Objekt der Klasse \texttt{DataSetItem} repräsentiert. Jedes Datum hat:
\begin{itemize}
    \item einen Namen (Zeichenkette),
    \item eine ID (Zahl),
    \item und beliebigen Inhalt.
\end{itemize}

Mehrere Daten, die Objekte vom Typ \texttt{DataSetItem} sind, werden in einem Datensatz zusammengefasst. Die Klasse \texttt{DataSetInterface} definiert die Schnittstelle und die Operationen, die ein Datensatz unterstützen muss. Ihre Aufgabe ist es, eine Klasse \texttt{DataSet} als Unterklasse von \texttt{DataSetInterface} zu implementieren.

\subsection{Aufbau}
Es gibt drei Dateien:
\begin{enumerate}
    \item \texttt{dataset.py}: Enthält die Klassen \texttt{DataSetInterface} und \texttt{DataSetItem}.
    \item \texttt{implementation.py}: Hier wird die Klasse \texttt{DataSet} implementiert.
    \item \texttt{main.py}: Nutzt die Klassen \texttt{DataSet} und \texttt{DataSetItem}, um die Schnittstelle und Operationen von \texttt{DataSetInterface} zu testen.
\end{enumerate}

\subsection{Methoden}
Die Klasse \texttt{DataSet} soll die folgenden Methoden implementieren (Details finden Sie in \texttt{dataset.py}):
\begin{itemize}
    \item \texttt{\_\_setitem\_\_(self, name, id\_content)}: Hinzufügen eines Datums mit Name, ID und Inhalt.
    \item \texttt{\_\_iadd\_\_(self, item)}: Hinzufügen eines \texttt{DataSetItem}.
    \item \texttt{\_\_delitem\_\_(self, name)}: Löschen eines Datums basierend auf seinem Namen.
    \item \texttt{\_\_contains\_\_(self, name)}: Prüfen, ob ein Datum mit diesem Namen im Datensatz vorhanden ist.
    \item \texttt{\_\_getitem\_\_(self, name)}: Abrufen eines Datums über seinen Namen.
    \item \texttt{\_\_and\_\_(self, dataset)}: Schnittmenge zweier Datensätze bestimmen und als neuen Datensatz zurückgeben.
    \item \texttt{\_\_or\_\_(self, dataset)}: Vereinigung zweier Datensätze bestimmen und als neuen Datensatz zurückgeben.
    \item \texttt{\_\_iter\_\_(self)}: Iteration über alle Daten des Datensatzes (optionale Sortierung).
    \item \texttt{filtered\_iterate(self, filter)}: Gefilterte Iteration über einen Datensatz mit einer Lambda-Funktion als Filter.
    \item \texttt{\_\_len\_\_(self)}: Anzahl der Daten im Datensatz abrufen.
\end{itemize}

% Abgabe
\section{Abgabe}
Implementieren Sie die Klasse \texttt{DataSet} in der Datei \texttt{implementation.py}. Sie können lokal arbeiten oder das VPL nutzen. Die Dateien \texttt{dataset.py}, \texttt{implementation.py} und \texttt{main.py} finden Sie im Ordner \texttt{code} des Git-Repositories.

Die Tests im VPL überprüfen Ihre Implementierung und helfen sicherzustellen, dass die richtigen Klassen genutzt werden.

\end{document}